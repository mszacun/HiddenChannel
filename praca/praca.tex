\documentclass[a4paper, twoside]{report}

\usepackage{titlesec}
\usepackage{polski}
\usepackage[utf8]{inputenc}
\usepackage[margin=25mm]{geometry}
\usepackage{indentfirst}
\linespread{1}
\titleformat*{\section}{\fontsize{14pt}{2}\bfseries}
\titleformat*{\subsection}{\fontsize{13pt}{2}\bfseries}
\titleformat*{\subsubsection}{\fontsize{13pt}{2}\bfseries}
\usepackage[T1]{fontenc}
\usepackage{mathptmx}



\begin{document}

\title{Ukryte kanały transmisji danych w sieciach komputerowych}
\author{Marcin Szachun}
\maketitle


\begin{abstract}
    Zapewnienie bezpiecznej komunikacji było od zawsze jednym z największych
    problemów telekomunikacji. Tradycyjnie bezpieczna transmisja była możliwa
    dzięki wykorzystaniu kryptografii. Wraz z zastosowanie coraz bardziej wysublimowanych
    ataków na algorytmy kryptograficzne, oprócz udoskonalania tych algorytmów rozpoczęto
    poszukiwania innych sposobów zabezpieczenia wymienianych informacji. Jednym
    z proponowanych rozwiązań jest steganografia, nauka zajmująca się opracowywaniem
    metod komunikacji, w sposób utrudniający wykrycie istnienia wymienianych komunikatów.
    Niniejsza praca zawiera krótki przegląd przykładowych sposobów
    realizacji kanałów ukrytych na potrzeby transmisji danych w sieciach komputerowych.
    W głównej części pracy wysuwana jest propozycja stworzenia kanału ukrytego, opartego na ukrywaniu
    danych w długości pakietów protokołu UDP przesyłanych kanałem podstawowym.
    Przedstawiony protokół kanału ukrytego wspiera zarówno agregację jak i fragmentację
    wiadomości podstawowych w celu transmisji wiadomości ukrytych.Przeprowadzone
    zostały szczegółowe badania parametrów jakościowych takiego kanału w zależności
    od przyjętych parametrów protokołu kanału ukrytego. Podjęta została również
    analiza ingerencji w kanał podstawowy oraz wykrywalności kanału zorganizowanego w taki sposób.
    Dokument zawiera wskazówki doboru parametrów protokołu kanału ukrytego, w taki sposób, aby
    zminimalizować jego ingerencję w kanał podstawowy. Ponadto zostały podane inne
    wskazówki pozwalające zmniejszyć szansę na wykrycie faktu, że kanał ukryty jest
    używany. Zaprezentowano również przykładowe środowiska, w których przedstawiony
    kanał ukryty może być wykorzystywany.
\end{abstract}

\tableofcontents

\chapter{Cel i zakres pracy}
    Celem niniejszej pracy jest przedstawienie zaprojektowanego protokołu kanału
    ukrytego, działającego w sieciach komputerowych i  opartego o ukrywanie wiadomości za pomocą długości pakietów UDP
    zawierających wiadomości podstawowe. W ramach tego celu zostaną wykonane:
    \begin{itemize}
        \item prezentacja steganografii, nauki o ukrywaniu wiadomości w różnego rodzaju nośnikach,
            oraz steganoanalizy, nauki o wykrywaniu oraz ewentualnym niszczeniu kanałów
            ukrytej komunikacji, wraz z najczęściej używanymi pojęciami
        \item przedstawienie przykładowych, innych niż zaproponowany w pracy,
            sposobów organizacji kanałów ukrytych, z zakresu steganografii sieciowej,
            z wykorzystaniem różnego rodzaju protokołów komunikacyjnych
        \item zaprezentowanie projektu protokołu kanału ukrytego, ukrywającego
            wiadomości ukryte za pomocą długości pakietów protokołu UDP, wraz
            z opisem parametrów otrzymanego protokołu, oraz wyjaśnieniem pojęć
            użytych do jego opisu
        \item przeprowadzenie badań czynników wpływających na średnie opóźnienie
            w odbiorze wiadomości podstawowych i ukrytych, wraz z analizą wniosków
            z nich płynących
        \item przeprowadzenie analizy wykrywalności zaproponowanego kanału ukrytego,
            wraz z zaproponowaniem sposobów jej zmniejszenia
    \end{itemize}

\chapter{Wstęp teoretyczny}
    \section{Steganografia}
        \subsection{Podstawowe pojęcia}
        \emph{Steganografia} jest to nauka o przesyłaniu komunikatów w taki sposób, aby
        osoba mająca dostęp do kanału, którym prowadzona jest transmisja komunikatów
        nie był w stanie zorientować się, że przesyłane komunikaty istnieją.
        Do ukrycia wiadomości ukrytych wykorzystuje się 
        \emph{nośniki steganograficzne}\cite{STEGANOGRAFIASIECIOWAART}.
        Jednymi z najczęściej wykorzystywanych nośników są multimedia takie jak
        obrazy oraz dźwięki. Poprzez modyfikację własności nośnika możliwe jest
        ukrycie w nim \emph{wiadomości ukrytej}. Pomimo, że osoba postronna ma
        dostęp do zmodyfikowanego nośnika z zawartą w nim wiadomością ukrytą,
        czyli tak zwanym \emph{steganogramem}, nie wzbudza on w niej podejrzeń,
        że może on przenosić jakieś ukryte dane. Procedurę mówiącą w jaki sposób
        zmodyfikować właściwości nośnika aby ukryć w nim wiadomość opisuje
        \emph{algorytm steganograficzny}. W ramach algorytmu steganograficznego
        opisuje się także przekształcenie odwrotne, które musi
        wykonać odbiorca steganogramu w celu odzyskania wiadomości ukrytej.
        Opcjonalnie zachowanie algorytmu steganograficznego może być modyfikowane przy pomocy specjalnego
        parametru, tak zwanego \emph{klucza}.  Dla takiej samej wiadomości ukrytej i nośnika
        wykorzystywanego do jej ukrycia, różne wartości klucza powodują, że algorytm
        w inny sposób modyfikuje właściwości nośnika, w rezultacie dając inne
        steganogramy.

        \subsection{Zastosowanie}
        Steganografia nie jest nową nauką. Pierwsze wzmianki o zastosowaniu technik
        służących do ukrywania informacji pochodzą ze starożytnej Grecji.\cite{STEGANOGRAPHYINTRO}
        W tamtych czasach wykorzystywane one były do przesyłania wiadomości
        podczas pobytu w niewoli, bądź w celu planowania różnego rodzaju spisków.

        Obecnie zainteresowanie steganografią rośnie i znajduje ona coraz więcej
        zastosowań w różnych dziedzinach. Podstawowym celem w jakim stosowana jest
        steganografia jest zabezpieczenie przed dostępem do wiadomości osób do tego nie
        upoważnionych. W większości przypadków, wykorzystywana jest do tego
        kryptografia. W odróżnieniu od steganografii, kryptografia nie stara się
        ukryć faktu istnienia wiadomości, tylko za pomocą algorytmów kryptograficznych
        przekształcić ją w taki sposób, aby nie możliwe było jej odczytanie bez posiadania
        klucza. Osoba, która przechwyci zaszyfrowaną komunikację nie może co prawda
        odczytać komunikatów, jednak jest dla niej oczywiste, że nadawca i odbiorca
        komunikują się ze sobą i że jakaś tajna wiadomość, nie przeznaczona dla niej,
        istnieje.\cite{DIGITALWATERMARKING} Cecha ta sprawia, że kryptografii nie
        powinno się stosować tam, gdzie konieczne jest zachowanie w tajemnicy
        faktu istnienia wiadomości lub faktu prowadzenia komunikacji. Jako przykład
        można tutaj podać komunikację pomiędzy różnego rodzaju służbami specjalnymi
        podczas organizacji i prowadzenia akcji specjalnych na przykład odbijanie zakładników.
        W przypadku gdyby przestępcy, przechwyciliby zaszyfrowaną komunikację
        pomiędzy oddziałami, mogliby nabrać podejrzeń, jeśli natomiast przechwycona
        zostałaby niezaszyfrowana niewinnie wyglądająca komunikacje, nie powinna ona
        budzić żadnych obaw. Steganografia wykorzystywana jest również do prowadzenia
        działań wywiadowczych i anty wywiadowczych. Szpiedzy wykorzystują ją do
        bezpiecznej i nie wzbudzającej podejrzeń komunikacji z macierzystą agencją
        wywiadowczą.

        Kryptografia nie może również zostać wykorzystana w miejscach, w których
        nie pozwala na to prawo.\cite{CRYPTOGRAFYLAW} W takich krajach wszelkie
        zaszyfrowane wiadomości mogą być niszczone przez instytucje zajmujące się
        cenzurą. Steganografia jest w takim
        wypadku jedynym środkiem umożliwiającym bezpieczną komunikację. Ponadto
        dzięki jej zastosowaniu, możliwe jest użycie zabronionych algorytmów szyfrujących,
        co dodatkowo zabezpiecza komunikację, ponieważ cenzor nie jest świadomy
        istnienia zaszyfrowanej wiadomości. W niektórych krajach użycie kryptografii
        może być dozwolone, jednak mogą być narzucone ograniczenia, na algorytmy
        kryptograficzne, które mogą być wykorzystywane, lub też na maksymalną
        długość klucza kryptograficznego. Może to rodzić podejrzenia, że ustawodawca
        specjalnie ogranicza dostępne metody kryptograficzne, aby w razie konieczności
        być w stanie złamać i odszyfrować wymieniane wiadomości. Ustawodawca
        może ograniczyć dostępne algorytmy, do własnościowych algorytmów, bądź
        udostępniać jedynie własnościowe implementacje algorytmów kryptograficznych.
        Powoduje to nieufność do udostępnionych rozwiązań, nie tylko ze względu
        na zasadę Kerckhoffsa zgodnie z którą:
        "Algorytm kryptograficzny nie powinien być utrzymywany w sekrecie i nie powinno
        być problemem wykradnięcie go przez wroga".\cite{KERCKHOS}, ale również,
        w świetle doniesień Edwarda Snowdena na temat progrmów takich jak
        Ballrun\cite{WIKI:BALLRUN}, budzi obawę o możliwość umieszczenia w tych
        algorytmach "tylnych furtek", umożliwiających odszyfrowanie zaszyfrowanych
        nim wiadomości.

        Kolejnym przykładem zastosowanie technik używanych w steganografii jest
        tworzenie znaków wodnych(ang. \emph{watermark}). Jest to dziedzina bardzo zbliżona do steganografii,
        jednak różnica polega na tym, że w przypadku steganografii w nośnej ukrywamy
        informację nie związaną bezpośrednio z samą nośną, natomiast w przypadku znakowania,
        ukryty komunikat w jakiś sposób odnosi się do nośnej. Przykładowym zastosowaniem
        znakowania jest umieszczenie informacji o autorze obrazu w pliku ze zdjęciem.
        Ponadto w przypadku znakowania, istnienie znaku wodnego, nie musi być tajemnicą,
        jednak nadal musi być on "ukryty", to znaczy być zawarty w nośnej, jednak
        nie przeszkadzać w jej odbiorze, na przykład w oglądaniu oznakowanego obrazu.
        Oprócz tego usunięcie znaku wodnego powinno być trudne dla osób do tego nie
        uprawnionych. W przypadku, gdyby druga osoba skopiowała dzieło innej osoby,
        w którym ukryty był znak wodny, może on posłużyć prawowitemu autorowi w
        sądzie do udowodnienia kto jest prawdziwym autorem.

        Bardzo zbliżonym do znakowania zastosowaniem technik steganograficznych jest
        tworzenie "odcisków palca"(ang. \emph{fingerprinting}). Jest to specjalny
        typ znaku wodnego, unikalny dla każdej kopii dzieła(pliku). Dzięki odciskowi
        palca możliwe jest zidentyfikowanie osoby winnej na przykład wycieku informacji
        do mediów, lub też nielegalnie udostępniającej plik w Internecie. W przypadku,
        gdy jeden z nabywców filmu udostępnia go nielegalnie w Internecie, sprzedawca,
        który w każdej sprzedanej kopii umieścił unikany odcisk palca, może porównać
        odcisk palca w udostępnianym pliku z zapisanymi wcześniej odciskami palców
        w kopiach sprzedanym poszczególnym klientom.

        Techniki ukrywania informacji wykorzystywane są także do rozbudowywania
        istniejących od dawna formatów danych, w celu przechowywania dodatkowych
        danych lub metadanych. Często zdarza się na przykład, że wraz z obrazem
        chcielibyśmy przechowywać jego opis. Dzięki steganografii możemy dodać
        taką zawartość nawet do formatów plików, które nie wspierają natywnie
        przechowywania tego typu danych, nie łamiąc wstecznej kompatybilności z innymi
        programami obsługującymi dany format plików. Przykładem takiego wykorzystania
        steganografii jest przechowywanie opisów zdjęć medycznych, na przykład
        z prześwietleń, w tym samym pliku co obraz.\cite{DISAPPEARINGCRYPTOEMBEDDINGMETDATA}

        \subsection{Przykładowe metody stegnograficzne}
        Jeden z pierwszych historyków greckich Herodot, w swoim dziele "Dzieje"
        opisuje historię starożytnego polityka grackiego Histiajosa, który spiskował
        ze swoim zięciem Arystagorasem, w celu wywołania powstania miast greckich
        przeciw Persom\cite{STEGANOGRAPHYINTRO}. Ponieważ Persowie nie ufali Histiajosowi, musiał on komunikować
        się ze swoim zięciem w sposób sekretny. Wedługo Herodota, wykorzystywał
        on w tym celu swojego najbardziej zaufanego niewolnika, najpierw goląc
        go na łyso, a następnie tatuując mu ukryte wiadomości na skórze głowy.
        Gdy niewolnikowi odrastały włosy, był on wysyłany do Arystagorasa, oficjalnie
        transportując inną, nie związaną ze spiskiem i niewinnie wyglądającą wiadomość.
        Gdy niewolnik docierał do miejsca przeznaczenia, spotykał się Arystagorasem i
        gdy nie było przy nich osób postronnych mówił on o istnieniu wiadomości ukrytej.
        Po ponownym ogoleniu głowy niewolnika Arystagoras zapoznawał się z wiadomością
        od teścia.

        Inną techniką ukrywania wiadomości opisywaną również przez Herodota
        jest wykorzystanie drewnianych tabliczek. Zazwyczaj tabliczki te pokrywane
        były woskiem, w którym można było utrwalić wiadomość, a następnie, gdy
        nie była ona już potrzebna, stopić wosk i otrzymać tabliczkę gotową do
        ponownego użycia. Grek Demaratos chcąc ostrzec Spartę przed atakiem Persów,
        wyrył ostrzeżenie bezpośrednio w drewnie, następnie pokrywające je woskiem
        i umieszczając w nim niewinnie brzmiącą wiadomość.

        W średniowieczu często stosowane były atramenty sympatyczne, czyli takie,
        które po zapisaniu wiadomości na papierze stawały się niewidoczne, a do
        odczytania ukrytej wiadomości konieczne było "wywołanie" dokumentu. Procedura
        wywoływania była różna w zależności od substancji, która została użyta jako
        atrament sympatyczny. Wśród substancji używanych w tym celu można wymienić:
        mleko, sok cytrynowy, mocz, ocet czy też amoniak. Atramenty sympatyczne
        były nadal wykorzystywane w czasie obu wojen światowych.

        Jedna z ciekawszych technik steganograficznych została zaprezentowana przez
        Gaspari Schotti\cite{NUTYSTEGANOGRAFIA},
        który zaproponował kodowanie liter za pomocą nut. Każdej literze alfabetu
        odpowiadała jedna nuta, różniąca się od innych pod względem wysokości
        dźwięku i czasu jego trwania. Dla osoby nie znającej się na muzyce, tak zakodowana
        wiadomość wygląda na zwykły utwór muzyczny, jednak gdyby zagrać ją na instrumencie
        muzycznym, najprawdopodobniej nie byłyby to miłe uchu dźwięki.

        W czasach, gdy w wielu krajach obowiązywała cenzura, powszechne było ukrywanie
        informacji w tekstach. Odbywało się to poprzez odpowiedni dobór słów,
        z których składał się tekst. Odczytywanie tak ukrytych wiadomości możliwe było
        dzięki na przykład odczytywaniu drugiej litery każdego ze słów, bądź pierwszej
        litery każdego ze zdań.

        Wspomniano wcześniej, że techniki steganograficzne wykorzystywane są do umieszczania
        odcisków palców w dokumentach. Warto tutaj przytoczyć sztuczkę zastosowaną
        przez Margaret Thatcher poirytowaną częstymi wyciekami zdjęć tajnych dokumentów
        do prasy. Zażądała ona przeprogramowania oprogramowania do redagowania tych
        dokumentów, w taki sposób, aby w każdej kopii udostępnionej jej współpracownikom,
        poprzez zastosowanie różnych odstępów pomiędzy wyrazami, ukryć informację
        komu udostępniono daną kopię. Dzięki temu, gdy ponownie pojawiły się one w
        prasie, mogła ona zidentyfikować współpracownika odpowiedzialnego za wyciek.\cite{DIGITALWATERMARKING}

        Wraz z rozwojem informatyki i komputerów pojawiły się nowe możliwości
        ukrywania danych. Pojawiło się zainteresowanie wykorzystaniem plików
        multimedialnych na przykład obrazów lub nagrań, jako nośnika dla wiadomości
        ukrytych. Wiąże się to z powstaniem dziedziny zwanej steganografią sieciową.
        Jednym z najpopularniejszych sposobów wykorzystywanych w tej dziedzinie
        jest ukrywanie danych poprzez modyfikację najmniej istotnego bitu. Jest
        to metoda wykorzystywana zarówno do plików graficznych\cite{LSBSTEGANGRAPHY}
        jak i dźwiękowych\cite{AUDIOLSBSTEGANGRAPHY}.
        Polega ona na modyfikacji najmniej znaczącego bitu opisującego kolor piksela
        (lub daną próbkę nagrania), w taki sposób, aby ukryć w nim część danych
        ukrytych i aby jednocześnie zmiana w wyglądzie(lub brzmieniu) obraz(filmu)
        była niedostrzegalna. Tak otrzymany steganogram może być łatwo rozpowszechniany
        w Internecie, a dostęp do niego może mieć duża grupa osób.

        Jednym z najnowszych obszarów steganografii cyfrowej jest steganografia
        sieciowa, której poświęcony został jeden z kolejnych rozdziałów pracy.
        Ponadto praca ta zawiera propozycję organizacji kanału ukrytego, zawierającą
        się właśnie w tym obszarze steganografii.

    \section{Steganoanaliza}
        \subsection{Podstawowe pojęcia}
        \subsection{Zastosowanie}
        * terroryzm
        \subsection{Przykłady zapobiegania wykorzystywaniu steganografii}

\chapter{Charakterystyka możliwości realizacji kanałów ukrytych w sieciach komputerowych}
    \section{Poprzez modyfikację pakietów}
        \subsection{Wykorzystanie nieużywanych pól nagłówków}
        \subsection{Wykorzystanie dopełnienia pakietów}
        \subsection{Ukrywanie danych w pakietach celowo uszkodzonych}

    \section{Poprzez modyfikację właściwości strumienia pakietów}
        \subsection{Poprzez manipulację prędkością transmisji}
        \subsection{Poprzez manipulację opóźnieniem pakietów}

    \section{Podejścia hybrydowe}
        \subsection{Ukrywanie danych w pakietach opóźnionych}


\chapter{Projekt protokołu kanału ukrytego}
    \section{Opis otoczenia działania protokołu}
    \section{Założenia i ograniczenia projektowe}
    \section{Słownik używanych pojęć}
    \section{Opis parametrów kanału ukrytego}
    \section{Opis działania protokołu kanału ukrytego}
    \section{Schemat działania protokołu kanału ukrytego}

\chapter{Badania optymalnych parametrów i jakości kanału ukrytego}
    \section{Zależność opóźnienia wiadomości ukrytych od natężenia napływu wiadomości podstawowych}
        \subsection{Metodologia i cel badania}
        \subsection{Obserwacje}
        \subsection{Wnioski}

    \section{Zależność opóźnienia wiadomości podstawowych od natężenia napływu wiadomości ukrytych}
        \subsection{Metodologia i cel badania}
        \subsection{Obserwacje}
        \subsection{Wnioski}

    \section{Zależność opóźnienia wiadomości podstawowych od natężenia napływu wiadomości podstawowych}
        \subsection{Metodologia i cel badania}
        \subsection{Obserwacje}
        \subsection{Wnioski}


\chapter{Analiza wykrywalności kanału ukrytego}
    \section{Analiza ingerencji kanału ukrytego w kanał podstawowy}
    \section{Metody ataku na kanał ukryty}
    \section{Sugestie doboru parametrów kanału ukrytego}
    \section{Inne sposoby obrony przed wykryciem}

\chapter{Podsumowanie i wnioski}
\chapter{Możliwości rozwoju i kontynuacji prac}

\clearpage
\addcontentsline{toc}{chapter}{Bibliografia}
\bibliographystyle{plain}
\bibliography{bibliografia}


\end{document}
