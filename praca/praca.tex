\documentclass[a4paper]{report}

\usepackage{titlesec}
\usepackage{polski}
\usepackage[utf8]{inputenc}
\usepackage[margin=25mm]{geometry}
\linespread{1}
\titleformat*{\section}{\fontsize{14pt}{2}\bfseries}
\titleformat*{\subsection}{\fontsize{13pt}{2}\bfseries}
\titleformat*{\subsubsection}{\fontsize{13pt}{2}\bfseries}
\usepackage[T1]{fontenc}
\usepackage{mathptmx}



\begin{document}

\title{Ukryte kanały transmisji danych w sieciach komputerowych}
\author{Marcin Szachun}
\maketitle


\begin{abstract}
To jest tekst
\end{abstract}

\tableofcontents

\chapter{Cel i zakres pracy}
\chapter{Wstęp teoretyczny}
    \section{Steganografia}
        \subsection{Podstawowe pojęcia}
        \subsection{Zastosowanie}
        \subsection{Przykłady dawnych zastosowań}

    \section{Steganografia}
        \subsection{Podstawowe pojęcia}
        \subsection{Zastosowanie}
        \subsection{Przykłady zapobiegania wykorzystywaniu steganografii}

\chapter{Charakterystyka możliwości realizacji kanałów ukrytych}
    \section{Poprzez modyfikację pakietów}
        \subsection{Wykorzystanie nieużywanych pól nagłówków}
        \subsection{Wykorzystanie dopełnienia pakietów}
        \subsection{Ukrywanie danych w pakietach celowo uszkodzonych}

    \section{Poprzez modyfikację właściwości strumienia pakietów}
        \subsection{Poprzez manipulację prędkością transmisji}
        \subsection{Poprzez manipulację opóźnieniem pakietów}

    \section{Podejścia hybrydowe}
        \subsection{Ukrywanie danych w pakietach opóźnionych}


\chapter{Projekt protokołu kanału ukrytego}
    \section{Opis otoczenia działania protokołu}
    \section{Założenia i ograniczenia projektowe}
    \section{Słownik używanych pojęć}
    \section{Opis parametrów kanału ukrytego}
    \section{Opis działania protokołu kanału ukrytego}
    \section{Schemat działania protokołu kanału ukrytego}

\chapter{Badania optymalnych parametrów i jakości kanału ukrytego}
    \section{Zależność opóźnienia wiadomości ukrytych od natężenia napływu wiadomości podstawowych}
        \subsection{Metodologia i cel badania}
        \subsection{Obserwacje}
        \subsection{Wnioski}

    \section{Zależność opóźnienia wiadomości podstawowych od natężenia napływu wiadomości ukrytych}
        \subsection{Metodologia i cel badania}
        \subsection{Obserwacje}
        \subsection{Wnioski}

    \section{Zależność opóźnienia wiadomości podstawowych od natężenia napływu wiadomości podstawowych}
        \subsection{Metodologia i cel badania}
        \subsection{Obserwacje}
        \subsection{Wnioski}


\chapter{Analiza wykrywalności kanału ukrytego}
    \section{Analiza ingerencji kanału ukrytego w kanał podstawowy}
    \section{Metody ataku na kanał ukryty}
    \section{Sugestie doboru parametrów kanału ukrytego}
    \section{Inne sposoby obrony przed wykryciem}

\chapter{Podsumowanie i wnioski}
\chapter{Możliwości rozwoju i kontynuacji prac}
\chapter{Bibliografia}


\end{document}
