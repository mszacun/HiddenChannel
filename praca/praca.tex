\documentclass[a4paper]{report}

\usepackage{titlesec}
\usepackage{polski}
\usepackage[utf8]{inputenc}
\usepackage[margin=25mm]{geometry}
\linespread{1}
\titleformat*{\section}{\fontsize{14pt}{2}\bfseries}
\titleformat*{\subsection}{\fontsize{13pt}{2}\bfseries}
\titleformat*{\subsubsection}{\fontsize{13pt}{2}\bfseries}
\usepackage[T1]{fontenc}
\usepackage{mathptmx}



\begin{document}

\title{Ukryte kanały transmisji danych w sieciach komputerowych}
\author{Marcin Szachun}
\maketitle


\begin{abstract}
    Zapewnienie bezpiecznej komunikacji było od zawsze jednym z największych
    problemów telekomunikacji. Tradycyjnie bezpieczna transmisja była możliwa
    dzięki wykorzystaniu kryptografii. Wraz z zastosowanie coraz bardziej wysublimowanych
    ataków na algorytmy kryptograficzne, oprócz udoskonalania tych algorytmów rozpoczęto
    poszukiwania innych sposobów zabezpieczenia wymienianych informacji. Jednym
    z proponowanych rozwiązań jest steganografia, nauka zajmująca się opracowywaniem
    metod komunikacji, w sposób utrudniający wykrycie istnienia wymienianych komunikatów.
    Niniejsza praca zawiera krótki przegląd przykładowych sposobów
    realizacji kanałów ukrytych na potrzeby transmisji danych w sieciach komputerowych.
    W głównej części pracy wysuwana jest propozycja stworzenia kanału ukrytego, opartego na ukrywaniu
    danych w długości pakietów protokołu UDP przesyłanych kanałem podstawowym.
    Przedstawiony protokół kanału ukrytego wspiera zarówno agregację jak i fragmentację
    wiadomości podstawowych w celu transmisji wiadomości ukrytych.Przeprowadzone
    zostały szczegółowe badania parametrów jakościowych takiego kanału w zależności
    od przyjętych parametrów protokołu kanału ukrytego. Podjęta została również
    analiza ingerencji w kanał podstawowy oraz wykrywalności kanału zorganizowanego w taki sposób.
    Dokument zawiera wskazówki doboru parametrów protokołu kanału ukrytego, w taki sposób, aby
    zminimalizować jego ingerencję w kanał podstawowy. Ponadto zostały podane inne
    wskazówki pozwalające zmniejszyć szansę na wykrycie faktu, że kanał ukryty jest
    używany. Zaprezentowano również przykładowe środowiska, w których przedstawiony
    kanał ukryty może być wykorzystywany.
\end{abstract}

\tableofcontents

\chapter{Cel i zakres pracy}
    Celem niniejszej pracy jest przedstawienie zaprojektowanego protokołu kanału
    ukrytego, działającego w sieciach komputerowych i  opartego o ukrywanie wiadomości za pomocą długości pakietów UDP
    zawierających wiadomości podstawowe. W ramach tego celu zostaną wykonane:
    \begin{itemize}
        \item prezentacja steganografii, nauki o ukrywaniu wiadomości w różnego rodzaju nośnikach,
            oraz steganoanalizy, nauki o wykrywaniu oraz ewentualnym niszczeniu kanałów
            ukrytej komunikacji, wraz z najczęściej używanymi pojęciami
        \item przedstawienie przykładowych, innych niż zaproponowany w pracy,
            sposobów organizacji kanałów ukrytych, z zakresu steganografii sieciowej,
            z wykorzystaniem różnego rodzaju protokołów komunikacyjnych
        \item zaprezentowanie projektu protokołu kanału ukrytego, ukrywającego
            wiadomości ukryte za pomocą długości pakietów protokołu UDP, wraz
            z opisem parametrów otrzymanego protokołu, oraz wyjaśnieniem pojęć
            użytych do jego opisu
        \item przeprowadzenie badań czynników wpływających na średnie opóźnienie
            w odbiorze wiadomości podstawowych i ukrytych, wraz z analizą wniosków
            z nich płynących
        \item przeprowadzenie analizy wykrywalności zaproponowanego kanału ukrytego,
            wraz z zaproponowaniem sposobów jej zmniejszenia
    \end{itemize}

\chapter{Wstęp teoretyczny}
    \section{Steganografia}
        \subsection{Podstawowe pojęcia}
        \emph{Steganografia} jest to nauka o przesyłaniu komunikatów w taki sposób, aby
        osoba mająca dostęp do kanału, którym prowadzona jest transmisja komunikatów
        nie był w stanie zorientować się, że przesyłane komunikaty istnieją.
        Do ukrycia wiadomości ukrytych wykorzystuje się \emph{nośniki steganograficzne}.
        Jednymi z najczęściej wykorzystywanych nośników są multimedia takie jak
        obrazy oraz dźwięki. Poprzez modyfikację własności nośnika możliwe jest
        ukrycie w nim \emph{wiadomości ukrytej}. Pomimo, że osoba postronna ma
        dostęp do zmodyfikowanego nośnika z zawartą w nim wiadomością ukrytą,
        czyli tak zwanym \emph{steganogramem}, nie wzbudza on w niej podejrzeń,
        że może on przenosić jakieś ukryte dane. Procedurę mówiącą w jaki sposób
        zmodyfikować właściwości nośnika aby ukryć w nim wiadomość opisuje
        \emph{algorytm steganograficzny}. Opisuje on także procedurę jaką musi
        wykonać odbiorca steganogramu w celu odzyskania wiadomości ukrytej.
        Opcjonalnie zachowanie algorytmu steganograficznego może być modyfikowane przy pomocy specjalnego
        parametru, tak zwanego \emph{klucza}.  Dla takiej samej wiadomości ukrytej i nośnika
        wykorzystywanego do jej ukrycia, różne wartości klucza powodują, że algorytm
        w inny sposób modyfikuje właściwości nośnika, w rezultacie dając inne
        steganogramy.

        \subsection{Zastosowanie}
        \subsection{Przykładowe metody stegnograficzne}

    \section{Steganoanaliza}
        \subsection{Podstawowe pojęcia}
        \subsection{Zastosowanie}
        \subsection{Przykłady zapobiegania wykorzystywaniu steganografii}

\chapter{Charakterystyka możliwości realizacji kanałów ukrytych w sieciach komputerowych}
    \section{Poprzez modyfikację pakietów}
        \subsection{Wykorzystanie nieużywanych pól nagłówków}
        \subsection{Wykorzystanie dopełnienia pakietów}
        \subsection{Ukrywanie danych w pakietach celowo uszkodzonych}

    \section{Poprzez modyfikację właściwości strumienia pakietów}
        \subsection{Poprzez manipulację prędkością transmisji}
        \subsection{Poprzez manipulację opóźnieniem pakietów}

    \section{Podejścia hybrydowe}
        \subsection{Ukrywanie danych w pakietach opóźnionych}


\chapter{Projekt protokołu kanału ukrytego}
    \section{Opis otoczenia działania protokołu}
    \section{Założenia i ograniczenia projektowe}
    \section{Słownik używanych pojęć}
    \section{Opis parametrów kanału ukrytego}
    \section{Opis działania protokołu kanału ukrytego}
    \section{Schemat działania protokołu kanału ukrytego}

\chapter{Badania optymalnych parametrów i jakości kanału ukrytego}
    \section{Zależność opóźnienia wiadomości ukrytych od natężenia napływu wiadomości podstawowych}
        \subsection{Metodologia i cel badania}
        \subsection{Obserwacje}
        \subsection{Wnioski}

    \section{Zależność opóźnienia wiadomości podstawowych od natężenia napływu wiadomości ukrytych}
        \subsection{Metodologia i cel badania}
        \subsection{Obserwacje}
        \subsection{Wnioski}

    \section{Zależność opóźnienia wiadomości podstawowych od natężenia napływu wiadomości podstawowych}
        \subsection{Metodologia i cel badania}
        \subsection{Obserwacje}
        \subsection{Wnioski}


\chapter{Analiza wykrywalności kanału ukrytego}
    \section{Analiza ingerencji kanału ukrytego w kanał podstawowy}
    \section{Metody ataku na kanał ukryty}
    \section{Sugestie doboru parametrów kanału ukrytego}
    \section{Inne sposoby obrony przed wykryciem}

\chapter{Podsumowanie i wnioski}
\chapter{Możliwości rozwoju i kontynuacji prac}

\clearpage
\addcontentsline{toc}{chapter}{Bibliografia}
\bibliographystyle{plain}
\bibliography{bibliografia}


\end{document}
