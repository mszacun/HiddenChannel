\documentclass[a4paper, twoside]{report}

\usepackage{titlesec}
\usepackage{polski}
\usepackage[utf8]{inputenc}
\usepackage[margin=25mm]{geometry}
\usepackage{indentfirst}
\linespread{1}
\titleformat*{\section}{\fontsize{14pt}{2}\bfseries}
\titleformat*{\subsection}{\fontsize{13pt}{2}\bfseries}
\titleformat*{\subsubsection}{\fontsize{13pt}{2}\bfseries}
\usepackage[T1]{fontenc}
\usepackage{mathptmx}



\begin{document}

\title{Ukryte kanały transmisji danych w sieciach komputerowych}
\author{Marcin Szachun}
\maketitle


\begin{abstract}
    Zapewnienie bezpiecznej komunikacji było od zawsze jednym z największych
    problemów telekomunikacji. Tradycyjnie bezpieczna transmisja była możliwa
    dzięki wykorzystaniu kryptografii. Wraz z zastosowanie coraz bardziej wysublimowanych
    ataków na algorytmy kryptograficzne, oprócz udoskonalania tych algorytmów rozpoczęto
    poszukiwania innych sposobów zabezpieczenia wymienianych informacji. Jednym
    z proponowanych rozwiązań jest steganografia, nauka zajmująca się opracowywaniem
    metod komunikacji, w sposób utrudniający wykrycie istnienia wymienianych komunikatów.
    Niniejsza praca zawiera krótki przegląd przykładowych sposobów
    realizacji kanałów ukrytych na potrzeby transmisji danych w sieciach komputerowych.
    W głównej części pracy wysuwana jest propozycja stworzenia kanału ukrytego, opartego na ukrywaniu
    danych w długości pakietów protokołu UDP przesyłanych kanałem podstawowym.
    Przedstawiony protokół kanału ukrytego wspiera zarówno agregację jak i fragmentację
    wiadomości podstawowych w celu transmisji wiadomości ukrytych.Przeprowadzone
    zostały szczegółowe badania parametrów jakościowych takiego kanału w zależności
    od przyjętych parametrów protokołu kanału ukrytego. Podjęta została również
    analiza ingerencji w kanał podstawowy oraz wykrywalności kanału zorganizowanego w taki sposób.
    Dokument zawiera wskazówki doboru parametrów protokołu kanału ukrytego, w taki sposób, aby
    zminimalizować jego ingerencję w kanał podstawowy. Ponadto zostały podane inne
    wskazówki pozwalające zmniejszyć szansę na wykrycie faktu, że kanał ukryty jest
    używany. Zaprezentowano również przykładowe środowiska, w których przedstawiony
    kanał ukryty może być wykorzystywany.
\end{abstract}

\tableofcontents

\chapter{Cel i zakres pracy}
    Celem niniejszej pracy jest przedstawienie zaprojektowanego protokołu kanału
    ukrytego, działającego w sieciach komputerowych i  opartego o ukrywanie wiadomości za pomocą długości pakietów UDP
    zawierających wiadomości podstawowe. W ramach tego celu zostaną wykonane:
    \begin{itemize}
        \item prezentacja steganografii, nauki o ukrywaniu wiadomości w różnego rodzaju nośnikach,
            oraz steganoanalizy, nauki o wykrywaniu oraz ewentualnym niszczeniu kanałów
            ukrytej komunikacji, wraz z najczęściej używanymi pojęciami
        \item przedstawienie przykładowych, innych niż zaproponowany w pracy,
            sposobów organizacji kanałów ukrytych, z zakresu steganografii sieciowej,
            z wykorzystaniem różnego rodzaju protokołów komunikacyjnych
        \item zaprezentowanie projektu protokołu kanału ukrytego, ukrywającego
            wiadomości ukryte za pomocą długości pakietów protokołu UDP, wraz
            z opisem parametrów otrzymanego protokołu, oraz wyjaśnieniem pojęć
            użytych do jego opisu
        \item przeprowadzenie badań czynników wpływających na średnie opóźnienie
            w odbiorze wiadomości podstawowych i ukrytych, wraz z analizą wniosków
            z nich płynących
        \item przeprowadzenie analizy wykrywalności zaproponowanego kanału ukrytego,
            wraz z zaproponowaniem sposobów jej zmniejszenia
    \end{itemize}

\chapter{Wstęp teoretyczny}
    \section{Steganografia}
        \subsection{Podstawowe pojęcia}
        \emph{Steganografia} jest to nauka o przesyłaniu komunikatów w taki sposób, aby
        osoba mająca dostęp do kanału, którym prowadzona jest transmisja komunikatów
        nie był w stanie zorientować się, że przesyłane komunikaty istnieją.
        Do ukrycia wiadomości ukrytych wykorzystuje się 
        \emph{nośniki steganograficzne}\cite{STEGANOGRAFIASIECIOWAART}.
        Jednymi z najczęściej wykorzystywanych nośników są multimedia takie jak
        obrazy oraz dźwięki. Poprzez modyfikację własności nośnika możliwe jest
        ukrycie w nim \emph{wiadomości ukrytej}. Pomimo, że osoba postronna ma
        dostęp do zmodyfikowanego nośnika z zawartą w nim wiadomością ukrytą,
        czyli tak zwanym \emph{steganogramem}, nie wzbudza on w niej podejrzeń,
        że może on przenosić jakieś ukryte dane. Procedurę mówiącą w jaki sposób
        zmodyfikować właściwości nośnika aby ukryć w nim wiadomość opisuje
        \emph{algorytm steganograficzny}. W ramach algorytmu steganograficznego
        opisuje się także przekształcenie odwrotne, które musi
        wykonać odbiorca steganogramu w celu odzyskania wiadomości ukrytej.
        Opcjonalnie zachowanie algorytmu steganograficznego może być modyfikowane przy pomocy specjalnego
        parametru, tak zwanego \emph{klucza}.  Dla takiej samej wiadomości ukrytej i nośnika
        wykorzystywanego do jej ukrycia, różne wartości klucza powodują, że algorytm
        w inny sposób modyfikuje właściwości nośnika, w rezultacie dając inne
        steganogramy.

        \subsection{Zastosowanie}
        Steganografia nie jest nową nauką. Pierwsze wzmianki o zastosowaniu technik
        służących do ukrywania informacji pochodzą ze starożytnej Grecji.\cite{STEGANOGRAPHYINTRO}
        W tamtych czasach wykorzystywane one były do przesyłania wiadomości
        podczas pobytu w niewoli, bądź w celu planowania różnego rodzaju spisków.

        Obecnie zainteresowanie steganografią rośnie i znajduje ona coraz więcej
        zastosowań w różnych dziedzinach. Podstawowym celem w jakim stosowana jest
        steganografia jest zabezpieczenie przed dostępem do wiadomości osób do tego nie
        upoważnionych. W większości przypadków, wykorzystywana jest do tego
        kryptografia. W odróżnieniu od steganografii, kryptografia nie stara się
        ukryć faktu istnienia wiadomości, tylko za pomocą algorytmów kryptograficznych
        przekształcić ją w taki sposób, aby nie możliwe było jej odczytanie bez posiadania
        klucza. Osoba, która przechwyci zaszyfrowaną komunikację nie może co prawda
        odczytać komunikatów, jednak jest dla niej oczywiste, że nadawca i odbiorca
        komunikują się ze sobą i że jakaś tajna wiadomość, nie przeznaczona dla niej,
        istnieje.\cite{DIGITALWATERMARKING} Cecha ta sprawia, że kryptografii nie
        powinno się stosować tam, gdzie konieczne jest zachowanie w tajemnicy
        faktu istnienia wiadomości lub faktu prowadzenia komunikacji. Jako przykład
        można tutaj podać komunikację pomiędzy różnego rodzaju służbami specjalnymi
        podczas organizacji i prowadzenia akcji specjalnych na przykład odbijanie zakładników.
        W przypadku gdyby przestępcy, przechwyciliby zaszyfrowaną komunikację
        pomiędzy oddziałami, mogliby nabrać podejrzeń, jeśli natomiast przechwycona
        zostałaby niezaszyfrowana niewinnie wyglądająca komunikacje, nie powinna ona
        budzić żadnych obaw. Steganografia wykorzystywana jest również do prowadzenia
        działań wywiadowczych i anty wywiadowczych. Szpiedzy wykorzystują ją do
        bezpiecznej i nie wzbudzającej podejrzeń komunikacji z macierzystą agencją
        wywiadowczą.

        Kryptografia nie może również zostać wykorzystana w miejscach, w których
        nie pozwala na to prawo.\cite{CRYPTOGRAFYLAW} W takich krajach wszelkie
        zaszyfrowane wiadomości mogą być niszczone przez instytucje zajmujące się
        cenzurą. Steganografia jest w takim
        wypadku jedynym środkiem umożliwiającym bezpieczną komunikację. Ponadto
        dzięki jej zastosowaniu, możliwe jest użycie zabronionych algorytmów szyfrujących,
        co dodatkowo zabezpiecza komunikację, ponieważ cenzor nie jest świadomy
        istnienia zaszyfrowanej wiadomości. W niektórych krajach użycie kryptografii
        może być dozwolone, jednak mogą być narzucone ograniczenia, na algorytmy
        kryptograficzne, które mogą być wykorzystywane, lub też na maksymalną
        długość klucza kryptograficznego. Może to rodzić podejrzenia, że ustawodawca
        specjalnie ogranicza dostępne metody kryptograficzne, aby w razie konieczności
        być w stanie złamać i odszyfrować wymieniane wiadomości. Ustawodawca
        może ograniczyć dostępne algorytmy, do własnościowych algorytmów, bądź
        udostępniać jedynie własnościowe implementacje algorytmów kryptograficznych.
        Powoduje to nieufność do udostępnionych rozwiązań, nie tylko ze względu
        na zasadę Kerckhoffsa zgodnie z którą:
        "Algorytm kryptograficzny nie powinien być utrzymywany w sekrecie i nie powinno
        być problemem wykradnięcie go przez wroga".\cite{KERCKHOS}, ale również,
        w świetle doniesień Edwarda Snowdena na temat programów takich jak
        Ballrun\cite{WIKI:BALLRUN}, budzi obawę o możliwość umieszczenia w tych
        algorytmach "tylnych furtek", umożliwiających odszyfrowanie zaszyfrowanych
        nim wiadomości.

        Kolejnym przykładem zastosowanie technik używanych w steganografii jest
        tworzenie znaków wodnych(ang. \emph{watermark}). Jest to dziedzina bardzo zbliżona do steganografii,
        jednak różnica polega na tym, że w przypadku steganografii w nośnej ukrywamy
        informację nie związaną bezpośrednio z samą nośną, natomiast w przypadku znakowania,
        ukryty komunikat w jakiś sposób odnosi się do nośnej. Przykładowym zastosowaniem
        znakowania jest umieszczenie informacji o autorze obrazu w pliku ze zdjęciem.
        Ponadto w przypadku znakowania, istnienie znaku wodnego, nie musi być tajemnicą,
        jednak nadal musi być on "ukryty", to znaczy być zawarty w nośnej, jednak
        nie przeszkadzać w jej odbiorze, na przykład w oglądaniu oznakowanego obrazu.
        Oprócz tego usunięcie znaku wodnego powinno być trudne dla osób do tego nie
        uprawnionych. W przypadku, gdyby druga osoba skopiowała dzieło innej osoby,
        w którym ukryty był znak wodny, może on posłużyć prawowitemu autorowi w
        sądzie do udowodnienia kto jest prawdziwym autorem.

        Bardzo zbliżonym do znakowania zastosowaniem technik steganograficznych jest
        tworzenie "odcisków palca"(ang. \emph{fingerprinting}). Jest to specjalny
        typ znaku wodnego, unikalny dla każdej kopii dzieła(pliku). Dzięki odciskowi
        palca możliwe jest zidentyfikowanie osoby winnej na przykład wycieku informacji
        do mediów, lub też nielegalnie udostępniającej plik w Internecie. W przypadku,
        gdy jeden z nabywców filmu udostępnia go nielegalnie w Internecie, sprzedawca,
        który w każdej sprzedanej kopii umieścił unikany odcisk palca, może porównać
        odcisk palca w udostępnianym pliku z zapisanymi wcześniej odciskami palców
        w kopiach sprzedanym poszczególnym klientom.

        Techniki ukrywania informacji wykorzystywane są także do rozbudowywania
        istniejących od dawna formatów danych, w celu przechowywania dodatkowych
        danych lub metadanych. Często zdarza się na przykład, że wraz z obrazem
        chcielibyśmy przechowywać jego opis. Dzięki steganografii możemy dodać
        taką zawartość nawet do formatów plików, które nie wspierają natywnie
        przechowywania tego typu danych, nie łamiąc wstecznej kompatybilności z innymi
        programami obsługującymi dany format plików. Przykładem takiego wykorzystania
        steganografii jest przechowywanie opisów zdjęć medycznych, na przykład
        z prześwietleń, w tym samym pliku co obraz.\cite{DISAPPEARINGCRYPTOEMBEDDINGMETDATA}

        \subsection{Przykładowe metody stegnograficzne}
        Jeden z pierwszych historyków greckich Herodot, w swoim dziele "Dzieje"
        opisuje historię starożytnego polityka grackiego Histiajosa, który spiskował
        ze swoim zięciem Arystagorasem, w celu wywołania powstania miast greckich
        przeciw Persom\cite{STEGANOGRAPHYINTRO}. Ponieważ Persowie nie ufali Histiajosowi, musiał on komunikować
        się ze swoim zięciem w sposób sekretny. Według Herodota, wykorzystywał
        on w tym celu swojego najbardziej zaufanego niewolnika, najpierw goląc
        go na łyso, a następnie tatuując mu ukryte wiadomości na skórze głowy.
        Gdy niewolnikowi odrastały włosy, był on wysyłany do Arystagorasa, oficjalnie
        transportując inną, nie związaną ze spiskiem i niewinnie wyglądającą wiadomość.
        Gdy niewolnik docierał do miejsca przeznaczenia, spotykał się Arystagorasem i
        gdy nie było przy nich osób postronnych mówił on o istnieniu wiadomości ukrytej.
        Po ponownym ogoleniu głowy niewolnika Arystagoras zapoznawał się z wiadomością
        od teścia.

        Inną techniką ukrywania wiadomości opisywaną również przez Herodota
        jest wykorzystanie drewnianych tabliczek. Zazwyczaj tabliczki te pokrywane
        były woskiem, w którym można było utrwalić wiadomość, a następnie, gdy
        nie była ona już potrzebna, stopić wosk i otrzymać tabliczkę gotową do
        ponownego użycia. Grek Demaratos chcąc ostrzec Spartę przed atakiem Persów,
        wyrył ostrzeżenie bezpośrednio w drewnie, następnie pokrywające je woskiem
        i umieszczając w nim niewinnie brzmiącą wiadomość.

        W średniowieczu często stosowane były atramenty sympatyczne, czyli takie,
        które po zapisaniu wiadomości na papierze stawały się niewidoczne, a do
        odczytania ukrytej wiadomości konieczne było "wywołanie" dokumentu. Procedura
        wywoływania była różna w zależności od substancji, która została użyta jako
        atrament sympatyczny. Wśród substancji używanych w tym celu można wymienić:
        mleko, sok cytrynowy, mocz, ocet czy też amoniak. Atramenty sympatyczne
        były nadal wykorzystywane w czasie obu wojen światowych.

        Jedna z ciekawszych technik steganograficznych została zaprezentowana przez
        Gaspari Schotti\cite{NUTYSTEGANOGRAFIA},
        który zaproponował kodowanie liter za pomocą nut. Każdej literze alfabetu
        odpowiadała jedna nuta, różniąca się od innych pod względem wysokości
        dźwięku i czasu jego trwania. Dla osoby nie znającej się na muzyce, tak zakodowana
        wiadomość wygląda na zwykły utwór muzyczny, jednak gdyby zagrać ją na instrumencie
        muzycznym, najprawdopodobniej nie byłyby to miłe uchu dźwięki.

        W czasach, gdy w wielu krajach obowiązywała cenzura, powszechne było ukrywanie
        informacji w tekstach. Odbywało się to poprzez odpowiedni dobór słów,
        z których składał się tekst. Odczytywanie tak ukrytych wiadomości możliwe było
        dzięki na przykład odczytywaniu drugiej litery każdego ze słów, bądź pierwszej
        litery każdego ze zdań.

        Wspomniano wcześniej, że techniki steganograficzne wykorzystywane są do umieszczania
        odcisków palców w dokumentach. Warto tutaj przytoczyć sztuczkę zastosowaną
        przez Margaret Thatcher poirytowaną częstymi wyciekami zdjęć tajnych dokumentów
        do prasy. Zażądała ona przeprogramowania oprogramowania do redagowania tych
        dokumentów, w taki sposób, aby w każdej kopii udostępnionej jej współpracownikom,
        poprzez zastosowanie różnych odstępów pomiędzy wyrazami, ukryć informację
        komu udostępniono daną kopię. Dzięki temu, gdy ponownie pojawiły się one w
        prasie, mogła ona zidentyfikować współpracownika odpowiedzialnego za wyciek.\cite{DIGITALWATERMARKING}

        Wraz z rozwojem informatyki i komputerów pojawiły się nowe możliwości
        ukrywania danych. Pojawiło się zainteresowanie wykorzystaniem plików
        multimedialnych na przykład obrazów lub nagrań, jako nośnika dla wiadomości
        ukrytych. Wiąże się to z powstaniem dziedziny zwanej steganografią sieciową.
        Jednym z najpopularniejszych sposobów wykorzystywanych w tej dziedzinie
        jest ukrywanie danych poprzez modyfikację najmniej istotnego bitu. Jest
        to metoda wykorzystywana zarówno do plików graficznych\cite{LSBSTEGANGRAPHY}
        jak i dźwiękowych\cite{AUDIOLSBSTEGANGRAPHY}.
        Polega ona na modyfikacji najmniej znaczącego bitu opisującego kolor piksela
        (lub daną próbkę nagrania), w taki sposób, aby ukryć w nim część danych
        ukrytych i aby jednocześnie zmiana w wyglądzie(lub brzmieniu) obraz(filmu)
        była niedostrzegalna. Tak otrzymany steganogram może być łatwo rozpowszechniany
        w Internecie, a dostęp do niego może mieć duża grupa osób.

        Jednym z najnowszych obszarów steganografii cyfrowej jest steganografia
        sieciowa, której poświęcony został jeden z kolejnych rozdziałów pracy.
        Ponadto praca ta zawiera propozycję organizacji kanału ukrytego, zawierającą
        się właśnie w tym obszarze steganografii.

    \section{Steganoanaliza}
        \subsection{Podstawowe pojęcia}
        Podobnie jak kryptografii towarzyszy kryptoanaliza, tak i steganografii
        towarzyszy nauka zwana \emph{steganoanalizą}. Zajmuje się ona badaniem
        nośników, pod kątem przenoszenia przez nie danych ukrytych, ponadto badane
        są sposoby niszczenia danych ukrytych. W kryptografii system uznaje się
        za \emph{skompromitowany}, gdy kryptologowi uda się odczytać zaszyfrowaną wiadomość.
        Dla steganografii poprzeczka jest podniesiona, do jej skompromitowania wystarczy,
        że dowiedzione zostanie, że wiadomość ukryta istnieje w nośnej, nie jest
        konieczne jej odczytanie. Jeśli ponadto atakującemu uda się odczytać wiadomość
        ukrytą, zwiększa to poziom kompromitacji systemu.

        Wyróżnia się dwie metody prowadzenie steganoanlizy: pasywną i aktywną.
        Celem metody pasywnej jest wykrycie istnienia wiadomości ukrytej. Każdy przekazywany
        przez nadawcę komunikat jest analizowany pod względem zawierania tajnej wiadomości.
        W przypadku, gdy taka wiadomość zostanie wykryta, komunikacja zostaje przerwana,
        w przeciwnym wypadku, wiadomość jest dostarczana do odbiorcy. Osoba lub system,
        który zajmuje się analizą przesyłanych wiadomości, nazwana jest \emph{agentem(ang. warden)}.
        Steganoanaliza pasywna dzieli się na dwie grupy, \emph{steganoanaliza kierowana(ang. targeted)}, gdy
        atakujący podejrzewa algorytm, lub grupę algorytmów steganograficznych,
        które zostały użyte do ukrycia informacji i jest w stanie dopasować do
        niego sposób prowadzenia analizy, oraz \emph{steganoanalizę ślepą(ang. blind)},
        gdy musi on stosować ogólne sposoby analizy, które są w stanie wykryć różnego
        rodzaju algorytmy steganograficzne. Po wykryciu obecności wiadomości ukrytej,
        agent może przeprowadzić \emph{steganografię śledczą, wywiadowczą(ang. forensic steganalysis)},
        w celu wydobycia treści wiadomości ukrytej. W takim przypadku nie musi on
        blokować komunikacji pomiędzy odbiorcą a nadawcą, co mogłoby spowodować,
        że zorientowali by się oni, że bezpieczeństwo ich komunikacji może być zagrożone.

        W przypadku prowadzenie \emph{steganoanalizy aktywnej} agent ma więcej możliwości
        działania. W szczególności może on modyfikować wiadomości przesyłane pomiędzy
        nadawcą a odbiorcą. Jego celem nie koniecznie musi być wykrycie, czy wiadomość
        ukryta istnieje, ale na przykład upewnienie się, że na pewno nie dotrze ona
        do odbiorcy. W związku z tym, nawet jeśli błędnie nie wykryje on wiadomości ukrytej,
        może on modyfikować wiadomość podstawową, starając się usunąć wiadomość ukrytą,
        ale nadal zachowując sens i znaczenie wiadomości podstawowej. Tego typu analiza
        najczęściej skierowana jest przeciwko omówionym wcześniej znakom wodnym,
        oraz cyfrowym odciskom palców.\cite[Rozdział 13]{DIGITALWATERMARKING}

        \subsection{Zastosowanie}
        Wraz ze wzrostem zainteresowania steganografią, steganoanaliza jest wykorzystywana
        coraz częściej, we wszystkich miejscach, gdzie podejrzewane jest stosowanie
        steganografii.

        Wspomniano wcześniej o krajach, które próbują ograniczać
        prawo do wolności słowa swoich obywateli, na przykład poprzez ograniczenia
        wykorzystania kryptografii. Wobec zagrożenia użyciem steganografii, cenzorzy,
        stosują steganoanalizę, w celu wykrycia prób sekretnej komunikacji i jej zapobiegnięciu.
        Ze względu na specyfikę działania cenzora, zazwyczaj używa on steganoanalizy
        aktywnej, aby upewnić się, że jakiekolwiek wiadomości ukryte zostaną zniszczone

        Steganoanliza jest powszechnie wykorzystywana w zastosowaniach militarnych.
        Historia wielu wojen pokazuje, że poznanie zawczasu planów, oraz pozycji
        wojsk przeciwnika, pozwala uzyskać nad nim znaczną przewagę. Jako przykład
        można tutaj przytoczyć załamanie wiadomości zaszyfrowanych niemiecką maszyną
        szyfrującą "Enigma", w którym znaczący udział mieli polscy matematycy i kryptolodzy.
        Szacuje się, że dzięki temu II Wojna Światowa trwała około 2 -3 lata krócej
        i ocalone zostało około 30 milinów ludzi. Poprzez niszczenie ukrytych wiadomości wroga,
        w konsekwencji zakłócając komunikację i powodując dezorganizację we wrogich oddziałach,
        również można pogorszyć jego sytuację.
        W związku z tym, wywiad wojskowy wykorzystuje zarówno techniki opracowane w
        ramach steganoanalizy pasywnej, wywiadowczej i aktywnej.

        Niestety możliwość wymiany wiadomości bez wzbudzania podejrzeń bywa nadużywana.
        Oprócz wspomnianych wcześniej legalnych zastosowań, steganografia może
        być wykorzystywana przez organizacje przestępcze oraz terrorystyczne. Podejrzewa
        się, że do zorganizowania ataków na wieże World Trade Center, 11 września 2001 roku,
        terroryści wykorzystywali metody powiązane ze staganografią, a obecnie wykorzystują
        je powszechnie\cite{TERRORISMANDSTEGANOGRAPHY}.
        Powoduje to, że służby zajmujące się utrzymaniem bezpieczeństwa wykorzystują
        steganoanalizę do łapanie przestępców, oraz zapobieganiu ich działaniom.

        Najpowszechniejszym zastosowaniem steganoanalizy, podobnie jak kryptoanalizy,
        jest analizowanie oraz ocenianie nowych i już istniejących algorytmów steganograficznych.
        Jak wcześniej wspomniano algorytmy te wykorzystywane są w wielu miejscach,
        a niekiedy bezpieczeństwo zapewniane przez te algorytmy jest kluczowe, jak
        na przykład przy przesyłaniu tajnych wiadomości wojskowych, czy też przy znakowaniu
        utworów. Powoduje to, że poznanie mocnych i słabych stron algorytmów steganograficznych
        jest bardzo ważne. W celu oceny algorytmu, pod uwagę branych jest kilka czynników
        Bezpieczeństwo, czyli trudność i ilość pracy jaką trzeba włożyć,
        w badanie steganogramu, aby odkryć istnienie ukrytej wiadomości,
        w ramach badania bezpieczeństwa, zazwyczaj sprawdza się, jak ciężko
        jest odczytać treść ukrytej wiadomości, po odkryciu jej istnienia.
        Pojemność(bądź przepływność, w przypadku steganoanalizy sieciowej), czyli
        ilość informacji, które jest w stanie ukryć algorytm, w pojedynczej jednostce
        nośnika(na przykład w jednym obrazie, lub jednym pakiecie sieciowym). Badane
        jest także jak trudne jest usunięcie informacji ukrytej w steganogramie, przy
        użyciu steganoanalizy aktywnej.

        \subsection{Przykłady zapobiegania wykorzystywaniu steganografii}
        Jak zostało wcześniej wspomniane, do wykrywania i zapobiegania wykorzystaniu stegangrafii
        wykorzystywane są różnego rodzaju techniki opracowane w ramach steganoanalizy.
        Część ataków na alagorytmy steganograficzne jest podobna do ataków na
        algorytmy kryptograficzne. Steganografia stara się zapewnić dodatkowe bezpieczeństwo
        poprzez ukrycie wiadomości, w związku z tym jest podatna na dodatkowe ataki.
        Podobnie jak w przypadku kryptografii, im większą ilością danych na temat
        algorytmu steganograficznego i wymienianych wiadomości posiada atakujący,
        tym groźniejsze i bardziej wyrafinowane ataki jest w stanie on przeprowadzić.
        Wśród możliwych ataków wyróżniamy\cite{DISAPPEARINGCRYPTOEMBEDDINGMETDATA}:
        \begin{itemize}
            \item atak ze znanym steganogramem - atakujący ma dostęp jedynie do
                steganogramu, ewentualnie pewnej ich kolekcji, przechwyconych poprzez
                podsłuchiwanie komunikacji, na tej podstawie podejmuje on decyzję,
                czy zawierają one wiadomość ukrytą, czy nie, analiza ślepa opiera
                się w tym przypadku o analizę statystyczną steganogramu i poszukiwaniu
                odchyleń od normy lub o badanie zgodności wymienianych wiadomości
                ze standardami(na przykład zgody z oficjalną specyfikacją formatu pliku,
                lub protokołu komunikacyjnego), możliwe jest rozszerzenie
                analizy o próbę odczytania zawartości wiadomości ukrytej, po stwierdzeniu
                jej istnienia
            \item atak ze znanym steganogramem i oryginalnym nośnikiem - w przypadku
                takiego ataku steganoanalityk jest w stanie łatwo wykryć istnienie
                wiadomości poprzez porównanie nośnika ze steganogramem, jednoznacznie
                potwierdza to istnienie wiadomości ukrytej, ponieważ z definicji
                każdy algorytm steganograficzny powoduje modyfikację właściwości
                nośnika, w przypadku tego ataku najczęściej wysiłki skupiają się
                na odczytaniu zawartości wiadomości ukrytej
            \item atak ze znajomością algorytmu steganograficznego - atak ten
                stosowany jest przeciwko algorytmom używającym klucza steganograficznego,
                w celu odczytania zawartości wiadomości ukrytych, klucz steganograficzny
                jest najczęściej ziarnem dla generatora liczb losowych, który z
                kolei kontroluje proces ukrywania wiadomości w nośniku, przeprowadzanie
                takiego ataku przeciw algorytmom nie używającym klucza kryptograficznego
                nie jest konieczne, ponieważ znając algorytm, atakujące może wykrywać i
                odczytywać tajne wiadomości
            \item atak niszczący ukryte wiadomości - jak wspomniano wcześniej, niekiedy
                celem atakującego nie jest wykrycie wiadomości ukrytej, ale upewnienie
                się, że nie dotrze ona do odbiorcy, w tym celu może on poddać steganogram
                procesom takim jak filtrowanie(obrazy, dźwięki, skuteczne przeciwko
                algorytmom ukrywającym dane w najmniej znaczącym bicie), kompresja(bardzo skuteczna
                metoda przeciwko większości algorytmów), konwersja do innego formatu
                bądź użycie innego protokołu komunikacyjnego(skuteczniejsze niż konwersja) lub
                przeformułowanie wiadomości nośnej(na przykład poprzez zapisanie tego samego tekstu
                innymi słowami)
            \item atak zastępujący ukrytą wiadomość - atak stosowany w celu zniszczenia
                wiadomości ukrytej, lub zastąpienia jej wiadomością przygotowaną przez atakującego,
                stosowany, gdy dysponuje on specyfikacją algorytmu steganograficznego,
                polega na zastosowaniu algorytmu ponownie, traktując steganogram jako nośnik
                nowej ukrytej wiadomości, większość algorytmów jest podatna na ten
                atak, ponieważ ukrywają one wiadomość atakującego w taki sam sposób,
                jak oryginalną wiadomość ukrytą, tym samym ją niszcząc
        \end{itemize}


\chapter{Charakterystyka możliwości realizacji kanałów ukrytych w sieciach komputerowych}
    Wraz z rozwojem i coraz powszechniejszym wykorzystaniem komputerów i sieci
    komputerowych, prężnie rozwija się dziedzina steganografii zwana steganografią
    sieciową. Do tworzenia kanałów umożliwiających transmisję wiadomości ukrytych
    wykorzystuje ona protokoły komunikacyjne i wpływa na ich działanie. W kontekście
    steganografii sieciowej warto wyróżnić zasługi polskich naukowców z Politechniki
    Warszawskiej, którzy wiele wnieśli w rozwój tej nauki\cite{STEGANOGRAFIASIECIOWAART}.

    Protokoły komunikacyjne są zazwyczaj
    precyzyjnie specyfikowane w dokumentach RFC, jednak nawet najdokładniejsza
    specyfikacja nie jest w stanie pokryć wszystkich przypadków brzegowych. Powoduje
    to, że w niektórych miejscach specyfikacja pozostawia swobodę implementacji protokołu,
    co w dalszej kolejności powoduje występowanie różnic pomiędzy konkretnymi
    implementacjami, na przykład pomiędzy różnymi systemami operacyjnymi, ponadto
    w niektórych systemach implementacja protokołów może być okrojona. Te czynnik
    powodują, że pojawia się okazja do nadużyć i wykorzystaniu niespójności do organizowania
    kanałów ukrytych.

    Mnogość istniejących protokołów pozwala na tworzenie różnorodnych protokołów
    kanałów ukrytych, dostosowanych do konkretnych warunków komunikacyjnych.
    Mogą być w tym celu wykorzystywane protokoły ze wszystkich warstw modelu
    ISO/OSI, chociaż wykorzystanie niektórych warstw może być trudne, lub wiązać
    się z dużym wysiłkiem(na przykład dla warstwy fizycznej lub warstwy danych -
    implementacja dedykowanych sterowników urządzeń sieciowych). Umożliwia to
    też budowę dużych systemów steganograficznych wykorzystujących jednocześnie
    kilka kanałów ukrytych zorganizowanych na poziomie różnych warstw modelu ISO/OSI.
    Zastosowanie kilku kanałów ukrytych stosowanych na przemian, może pomóc w zmniejszeniu
    wykrywalności całego systemu poprzez minimalizowanie wywieranego wpływu na pojedynczy
    nośnik(protokół komunikacyjny).

    W dalszej części rozdziału przedstawiono podział steganografii sieciowej
    oraz zaprezentowane przykładowe kanały ukryte charakterystyczne dla danej
    dziedziny.
    \section{Poprzez modyfikację pakietów}
        \subsection{Wykorzystanie nieużywanych pól nagłówków}
        \subsection{Wykorzystanie dopełnienia pakietów}
        \subsection{Ukrywanie danych w pakietach celowo uszkodzonych}

    \section{Poprzez modyfikację właściwości strumienia pakietów}
        \subsection{Poprzez manipulację prędkością transmisji}
        \subsection{Poprzez manipulację opóźnieniem pakietów}

    \section{Podejścia hybrydowe}
        \subsection{Ukrywanie danych w pakietach opóźnionych}


\chapter{Projekt protokołu kanału ukrytego}
    \section{Opis otoczenia działania protokołu}
    \section{Założenia i ograniczenia projektowe}
    \section{Słownik używanych pojęć}
    \section{Opis parametrów kanału ukrytego}
    \section{Opis działania protokołu kanału ukrytego}
    \section{Schemat działania protokołu kanału ukrytego}

\chapter{Badania optymalnych parametrów i jakości kanału ukrytego}
    \section{Zależność opóźnienia wiadomości ukrytych od natężenia napływu wiadomości podstawowych}
        \subsection{Metodologia i cel badania}
        \subsection{Obserwacje}
        \subsection{Wnioski}

    \section{Zależność opóźnienia wiadomości podstawowych od natężenia napływu wiadomości ukrytych}
        \subsection{Metodologia i cel badania}
        \subsection{Obserwacje}
        \subsection{Wnioski}

    \section{Zależność opóźnienia wiadomości podstawowych od natężenia napływu wiadomości podstawowych}
        \subsection{Metodologia i cel badania}
        \subsection{Obserwacje}
        \subsection{Wnioski}


\chapter{Analiza wykrywalności kanału ukrytego}
    \section{Analiza ingerencji kanału ukrytego w kanał podstawowy}
    \section{Metody ataku na kanał ukryty}
    \section{Sugestie doboru parametrów kanału ukrytego}
    \section{Inne sposoby obrony przed wykryciem}

\chapter{Podsumowanie i wnioski}
\chapter{Możliwości rozwoju i kontynuacji prac}

\clearpage
\addcontentsline{toc}{chapter}{Bibliografia}
\bibliographystyle{plain}
\bibliography{bibliografia}


\end{document}
